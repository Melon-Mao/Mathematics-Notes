\documentclass[a4paper]{article}

\usepackage{amsmath}
\usepackage{amssymb}
\usepackage{amsfonts}
\usepackage{hyperref}
\usepackage[style=iso]{datetime2}
\usepackage{xcolor}
\usepackage{polynom}
\usepackage{tocloft}

\begin{titlepage}
\title{STEP Notes (2011.02.07)}
\author{Abdul Musthakin}
\date{\today}
\end{titlepage}

\renewcommand{\thesection}{\Roman{section}}

% \counterwithout{subsection}{section}  % have subsections numbered independently of sections
% \renewcommand{\thesubsection}{\arabic{subsection}.}  % have dot after subsection number

\addtolength{\cftsecnumwidth}{10pt}

\allowdisplaybreaks

\setlength{\parindent}{0pt}
\setlength{\parskip}{1em}

\begin{document}
\maketitle

I am typesetting my notes for this STEP problem, because I find it rather interesting.
Additionally, it is one of the first STEP problems that I have attempted and completed myself -- with almost no aid.
It is a geometric progressions' problem, and a fairly difficult one.
However, I found it simpler than most of the problems that I have seen far.
As I attempt more problems, I will get better at learning how to approach them.
Time-Management skills are not as important now, but they will be developed within due time.

Firstly, it is useful to express the formula for the sum of the first $n$ terms of a geometric progression.
\begin{equation}
    \sum_{i=1}^n ar^{i-1} = a + ar + ar^2 + \cdots + ar^n = \frac{a(1-r^n)}{1-r} \label{Geometric Series}
\end{equation}

We are given some information, all of which is useful.
Two sequences, $a_0, a_1, a_2, \ldots$, and $b_0, b_1, b_2, \ldots$ have the general terms,
\begin{equation*}
    a_n = \lambda^n + \mu^n \quad \text{and} \quad b_n = \lambda^n - \mu^n,
\end{equation*}
where
\begin{equation*}
    \lambda = 1 + \sqrt{2} \quad \text{and} \quad \mu = 1 - \sqrt{2}.
\end{equation*}

Part (i) provides us with a result which we must show is true.
\begin{equation}
    \sum_{r=0}^{n} b_r = -\sqrt{2} + \frac{1}{\sqrt{2}} a_{n+1}
\end{equation}
Additionally, it asks us to give a corresponding result for:
\begin{equation*}
    \sum_{r=0}^{n} a_r
\end{equation*}

Given the information we have, we can write the sum of $b_r$ in terms of $\lambda$ and $\mu$.
\begin{equation*}
    \sum_{r=0}^{n} b_r = \sum_{r=0}^{n} \lambda^r - \mu^r = \sum_{r=0}^{n} \lambda^r - \sum_{r=0}^{n} \mu^r
\end{equation*}
Now we have the difference between two geometric series; both of them can be evaluated using \eqref{Geometric Series}.
The initial term for both sums is 1 ($\lambda^0=\mu^0=1$), and the common ratios are $\lambda$ and $\mu$ respectively.
\begin{align}
    \sum_{i=0}^{n} \lambda^i = \sum_{i=1}^{n} \lambda^{i-1} = \frac{1-\lambda^n}{1-\lambda} + \lambda^n \\
    \sum_{i=0}^{n} \mu^i = \sum_{i=1}^{n} \mu^{i-1} = \frac{1-\mu^n}{1-\mu} + \mu^n
\end{align}

With these, we can find the sum of $b_r$.
It is better to not write $\lambda$ and $\mu$ as the expressions they were given as for the sake of keeping our work simple.
Those definitions will be utilised to simplify when possible.
\begin{align*}
    \sum_{r=0}^{n} b_r & = \frac{1-\lambda^n}{1-\lambda} - \frac{1-\mu^n}{1-\mu} + \lambda^n - \mu^n                                      \\
                       & = \frac{(1-\lambda^n)(1-\mu) - (1-\mu^n)(1-\lambda)}{(1-\lambda)(1-\mu)} + \lambda^n - \mu^n                     \\
                       & = \frac{(1-\lambda^n)(1-1+\sqrt{2}) - (1-\mu^n)(1-1-\sqrt{2})}{(1-1-\sqrt{2})(1-1+\sqrt{2})} + \lambda^n - \mu^n \\
                       & = \frac{\sqrt{2}(1-\lambda^n) + \sqrt{2}(1-\mu^n)}{-2} + \lambda^n - \mu^n                                       \\
                       & = -\frac{\sqrt{2}}{2}\bigg(1-\lambda^n+1-\mu^n\bigg) + \lambda^n - \mu^n                                         \\
                       & = -\frac{1}{\sqrt{2}}\bigg(2-\lambda^n-\mu^n\bigg) + \lambda^n - \mu^n                                           \\
                       & = -\sqrt{2} + \frac{1}{\sqrt{2}}\bigg(\lambda^n+\mu^n\bigg) + \lambda^n - \mu^n                                  \\
                       & = -\sqrt{2} + \frac{1}{\sqrt{2}}\bigg(\lambda^n+\mu^n + \sqrt{2}\lambda^n - \sqrt{2}\mu^n\bigg)                  \\
                       & = -\sqrt{2} + \frac{1}{\sqrt{2}}\bigg(\left(1+\sqrt{2}\right)\lambda^n + \left(1-\sqrt{2}\right)\mu^n\bigg)      \\
                       & = -\sqrt{2} + \frac{1}{\sqrt{2}}\bigg(\lambda\lambda^n + \mu\mu^n\bigg)                                          \\
                       & = -\sqrt{2} + \frac{1}{\sqrt{2}}\bigg(\lambda^{n+1} + \mu^{n+1}\bigg)                                            \\
                       & = -\sqrt{2} + \frac{1}{\sqrt{2}}a_{n+1} \qquad \square
\end{align*}

Now, we may do the same for the sum of $a_r$.
\begin{align*}
    \sum_{r=0}^{n} a_r & = \sum_{r=0}^{n} \lambda^r + \sum_{r=0}^{n} \mu^r                                                \\
                       & = \frac{1-\lambda^n}{1-\lambda} + \frac{1-\mu^n}{1-\mu} + \lambda^n + \mu^n                      \\
                       & = \frac{(1-\lambda^n)(1-\mu) + (1-\mu^n)(1-\lambda)}{(1-\lambda)(1-\mu)} + \lambda^n + \mu^n     \\
                       & = \frac{\sqrt{2}(1-\lambda^n) - \sqrt{2}(1-\mu^n)}{-2} + \lambda^n + \mu^n                       \\
                       & = -\frac{\sqrt{2}}{2}\bigg(1-\lambda^n - 1 + \mu^n\bigg) + \lambda^n + \mu^n                     \\
                       & = \frac{1}{\sqrt{2}}\bigg(\lambda^n - \mu^n\bigg) + \lambda^n + \mu^n                            \\
                       & = \frac{1}{\sqrt{2}}\bigg(\lambda^n - \mu^n + \sqrt{2}\lambda^n + \sqrt{2}\mu^n \bigg)           \\
                       & = \frac{1}{\sqrt{2}}\bigg(\left(1+\sqrt{2}\right)\lambda^n - \left(1-\sqrt{2}\right)\mu^n \bigg) \\
                       & = \frac{1}{\sqrt{2}}\bigg(\lambda^{n+1} - \mu^{n+1} \bigg)                                       \\
                       & = \frac{1}{\sqrt{2}}b_{n+1}
\end{align*}

So far, the most difficult part of the problem has been figuring out how to start it.
I needed to realize that I could write my sum as the difference between two geometric series, and the algebra that followed was relatively simple.
Additionally, I had to not make the mistake of expanding $\lambda$ and $\mu$; this would have made everything much more tedious.
As is characteristic of many STEP questions, I was provided with the result and told to show that it is true.
This allowed me to work backwards, which was a crucial strategy to figure out how to manipulate my expression to obtain the given result.

Finding the sum of $a_r$ was much simpler, as we already knew what to do and had the knowledge to simplify as was needed.
Our result allows us to complete the next part of the problem, which would be daunting otherwise.
Part (ii) requires the evaluation of a double summation.
A result is given for odd values of $n$.
\begin{equation}
    \sum_{m=0}^{2n} \left(\sum_{r=0}^{m} a_r\right) = \frac{1}{2} b^2_{n+1}
\end{equation}
The double sum can be turned into a single sum very easily -- just by substituting our result from part (i).
The rest of it is just algebraic manipulation, except for one thing.
We were told to find a result for the sum specifically for odd values of $n$.
This suggests that the parity of $n$ will be involved at some stage of our solution.
It was fairly easy to recognize where this was.

My simplification caused a $-1$ term to appear in the expression, which I rewrote as $\lambda\mu$.
This was something that was just natural to me.
I had already factorized terms to simplify my expression in part (i), and it is something I do fairly often in maths problems.
From that perspective, $\lambda\mu$ is much more useful to me than some constant term (when none of of my other terms are constants).
When raising $\lambda\mu$ to the power of $n$, the parity of $n$ is important.
\begin{equation}
    \lambda\mu^n = (-1)^n =
    \begin{cases}
        -1 = \phantom{-} \lambda\mu  & \text{if } x \text{ is odd}  \\
        \phantom{-}1  =  -\lambda\mu & \text{if } x \text{ is even}
    \end{cases}
\end{equation}

This is all we need to complete part (ii).
\begin{align*}
    \sum_{m=0}^{2n} \left(\sum_{r=0}^{m} a_r\right) & = \frac{1}{\sqrt{2}} \sum_{m=0}^{2n} b_{m+1}                                                                                                                                                   \\
                                                    & = \frac{1}{\sqrt{2}} \sum_{m=0}^{2n} (\lambda^m - \mu^m) + \sqrt{2} (\lambda^m + \mu^m)                                                                                                        \\
                                                    & = \frac{1}{\sqrt{2}} \sum_{m=0}^{2n} b_m + \sqrt{2} a_m                                                                                                                                        \\
                                                    & = \frac{1}{\sqrt{2}} \left( -\sqrt{2} + \frac{1}{\sqrt{2}} a_{2n+1} + \sqrt{2} \left(\frac{1}{\sqrt{2}} b_{2n+1} \right) \right)                                                               \\
                                                    & = -1 + \frac{1}{2} a_{2n+1} + \frac{1}{\sqrt{2}} b_{2n+1}                                                                                                                                      \\
                                                    & = -1 + \frac{1}{2} \lambda^{2n+1} + \frac{1}{2} \mu^{2n+1} + \frac{1}{\sqrt{2}} \lambda^{2n+1} - \frac{1}{\sqrt{2}} \mu^{2n+1}                                                                 \\
                                                    & = \lambda\mu + \frac{1}{2} \lambda\lambda^{2n} + \frac{1}{2} \mu\mu^{2n} + \frac{1}{\sqrt{2}} \lambda\lambda^{2n} - \frac{1}{\sqrt{2}} \mu\mu^{2n}                                             \\
                                                    & = \lambda\mu + \frac{1}{2} \lambda^{2n} + \frac{1}{2} \mu^{2n} + \frac{1}{\sqrt{2}} \lambda^{2n} - \frac{1}{\sqrt{2}} \mu^{2n} + \frac{1}{\sqrt{2}} \lambda^{2n} - \frac{1}{\sqrt{2}} \mu^{2n} \\
                                                    & \phantom{++}\quad + \lambda^{2n} + \mu^{2n}                                                                                                                                                    \\
                                                    & = \lambda\mu + \frac{1}{2} \bigg(\lambda^n - \mu^n \bigg)^2 + \lambda^n\mu^n + \frac{1}{\sqrt{2}} \bigg(\lambda^n + \mu^n \bigg) \bigg(\lambda^n - \mu^n \bigg)                                \\
                                                    & \phantom{++}\quad + \frac{1}{\sqrt{2}} \bigg(\lambda^n + \mu^n \bigg) \bigg(\lambda^n - \mu^n \bigg) + (\lambda^n + \mu^n)^2 - 2\lambda^n\mu^n                                                 \\
                                                    & = \lambda\mu - \lambda^n\mu^n + \frac{1}{2} \bigg(\lambda^n - \mu^n \bigg)^2 + \sqrt{2} (\lambda^n + \mu^n)(\lambda^n - \mu^n)                                                                 \\
                                                    & \phantom{++}\quad + (\lambda^n + \mu^n)^2                                                                                                                                                      \\
                                                    & = \lambda\mu - \lambda^n\mu^n + \frac{1}{2} b^2_n + \sqrt{2} a_n b_n + a^2_n                                                                                                                   \\
                                                    & = \frac{1}{2} b^2_n + \sqrt{2} a_n b_n + a^2_n                                                                                                                                                 \\
                                                    & = \frac{1}{2} \bigg(b^2_n + 2\sqrt{2} a_n b_n + 2a^2_n \bigg)                                                                                                                                  \\
                                                    & = \frac{1}{2} \bigg(b_n + \sqrt{2} a_n \bigg)^2                                                                                                                                                \\
                                                    & = \frac{1}{2} b^2_{n+1}, \text{ if $n$ is odd}. \qquad \square
\end{align*}

We can do this for even values of $n$, and most of our work can be reused.
\begin{align*}
    \sum_{m=0}^{2n} \left(\sum_{r=0}^{m} a_r\right) & = \lambda\mu - \lambda^n\mu^n + \frac{1}{2} b^2_n + \sqrt{2} a_n b_n + a^2_n \\
                                                    & = \frac{1}{2} b^2_n + \sqrt{2} a_n b_n + a^2_n + 2\lambda\mu                 \\
                                                    & = \frac{1}{2} b^2_{n+1} - 2, \text{ if $n$ is even}.
\end{align*}

A case could be made for part (ii) being easier than part (i).
The main difficulty in the problem is being able to start it.
There certainly is a lot of algebraic manipulation that needs to be done -- to the extent that one may get lost in it.
However, with the aid of being given the final result, the problem is fairly simpler conceptually.
Initially, when I was trying to see how I would obtain the final result, I expanded $b^2_{n+1}$.
This showed me how to factor the terms I had in my expression.
A goal to work towards is very useful with problems, and being able to work backwards was an essential strategy of mine.

Once I evaluated the sum for odd values of $n$, obtaining a result for even values was trivial.
I could see where the parity of $n$ would cause my work to diverge, with everything before it beign reusable.
Part (iii) follows the same formula as the first two parts in this regard.
We are asked to show that a result is true for odd values of $n$.
\begin{equation}
    \left(\sum_{r=0}^{n} a_r \right)^2 - \sum_{r=0}^{n} a_{2r+1} = 2
\end{equation}
I -- almost instantly -- thought of a method way to approach this.
The first term can easily be evaluated, as we have already found out what the sum is; squaring it is easy.
Then, we only have one sum to evaluate.
My idea was to use the given result to try and reverse-engineer the solution.
This was unnsuccessful, as I could not find a way to evaluate the sum just using the results that I had already obtained.

This was the only part of the problem that I obtained aid with.
I looked at part of the solution for this, just to see that the geometric series formula was to be used.
With this knowledge, I evaluated the sum pretty easily.
\begin{align*}
    \sum_{i=0}^{n} a_{2i+1} & = \sum_{i=0}^{n} \lambda^{2i+1} + \sum_{i=0}^{n} \mu^{2i+1}                                                                                                                                        \\
                            & = \sum_{i=0}^{n} \lambda\left(\lambda^2\right)^i + \sum_{i=0}^{n} \mu\left(\mu^2\right)^i                                                                                                          \\
                            & = \sum_{i=1}^{n} \lambda\left(\lambda^2\right)^{i-1} + \sum_{i=1}^{n} \mu\left(\mu^2\right)^{i-1} + \lambda^{2n+1} + \mu^{2n+1}                                                                    \\
                            & = \frac{\lambda\left(1-\lambda^{2n}\right)}{1-\lambda^2} + \frac{\mu\left(1-\mu^{2n}\right)}{1-\mu^2} + \lambda^{2n+1} + \mu^{2n+1}                                                                \\
                            & = \frac{\lambda-\lambda^{2n+1}}{1-\lambda^2} + \frac{\mu-\mu^{2n+1}}{1-\mu^2} + \lambda^{2n+1} + \mu^{2n+1}                                                                                        \\
                            & = \frac{\left(\lambda-\lambda^{2n+1}\right)\left(1-\mu^2\right) + \left(\mu-\mu^{2n+1}\right)\left(1-\lambda^2\right)}{\left(1-\lambda^2\right)\left(1-\mu^2\right)} + \lambda^{2n+1} + \mu^{2n+1} \\
                            & = \frac{\left(\lambda-\lambda^{2n+1}\right)\left(2\sqrt{2}-2\right) + \left(\mu-\mu^{2n+1}\right)\left(-2\sqrt{2}-2\right)}{\left(-2\sqrt{2}-2\right)\left(2\sqrt{2}-2\right)}                     \\
                            & \phantom{=}+ \lambda^{2n+1} + \mu^{2n+1}                                                                                                                                                           \\
                            & = \frac{-2\mu\left(\lambda-\lambda^{2n+1}\right) - 2\lambda\left(\mu-\mu^{2n+1}\right)}{4\lambda\mu} + \lambda^{2n+1} + \mu^{2n+1}                                                                 \\
                            & = -\frac{1}{2}\bigg(1 - \lambda^{2n}\bigg) - \frac{1}{2}\bigg(1 - \mu^{2n}\bigg) + \lambda^{2n+1} + \mu^{2n+1}                                                                                     \\
                            & = -1 + \frac{1}{2} \lambda^{2n} + \frac{1}{2} \mu^{2n} + \lambda\lambda^{2n} + \mu\mu^{2n}                                                                                                         \\
                            & = \lambda\mu + \frac{1}{2} \lambda^{2n} + \frac{1}{2} \mu^{2n} + \sqrt{2}\lambda^{2n} - \sqrt{2}\mu^{2n} + \lambda^{2n} + \mu^{2n}                                                                 \\
                            & = \frac{1}{2} b^2_n + \sqrt{2} a_n b_n + a^2_n - \lambda^n\mu^n + \lambda\mu                                                                                                                       \\
                            & =
    \begin{cases}
        \frac{1}{2} b^2_{n+1}    & \text{if } x \text{ is odd}  \\
        \frac{1}{2} b^2_{n+1} -2 & \text{if } x \text{ is even}
    \end{cases}
\end{align*}

There is no difficulty in completing the problem after this result is obtained.
For even values of $n$:
\begin{align*}
    \left(\sum_{r=0}^{n} a_r \right)^2 - \sum_{r=0}^{n} a_{2r+1} & = \left(\frac{1}{\sqrt{2}} b_{n+1}\right)^2 - \left(\frac{1}{2} b^2_{n+1} - 2\right) \\
                                                                 & = \frac{1}{2} b^2_{n+1} - \frac{1}{2} b^2_{n+1} + 2                                  \\
                                                                 & = 2 \qquad \square
\end{align*}
For odd $n$:
\begin{equation*}
    \left(\sum_{r=0}^{n} a_r \right)^2 - \sum_{r=0}^{n} a_{2r+1}  = \frac{1}{2} b^2_{n+1} - \frac{1}{2} b^2_{n+1} = 0
\end{equation*}

Overall, this is not a very difficult problem conceputally.
The algebraic manipulations are not incredibly difficult to perform, though it is a step above A-level.
The important thing is to realize how everything is interconnected.
With the same general method of evaluating geometric series, we can do everything we were asked to do.
There was the aspect of utilising the parity of a vairable to simplify our expression, though spotting is not difficult with prior experience of dealing with such problems.

Just as any other other problem I could have done, completing this and explaining my methadology, has improved my ability in general.
The way I did things is not exactly the same as the solutions I have seen.
It is not the most optimal way, and one can obtain the same results with less lines of work.
However, what I did was what made sense to me -- what came to my mind first.
My notes display my reasoning and the way I do things.
Whilst I will always attempt to improve them, being able to succeed with what I know and how I already think is a good sign.

\end{document}