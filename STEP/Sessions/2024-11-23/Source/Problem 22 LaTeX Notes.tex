\documentclass[12pt]{article}

\usepackage{amsmath}
\usepackage{amssymb}
\usepackage{amsfonts}
\usepackage{hyperref}
\usepackage[style=iso]{datetime2}
\usepackage{xcolor}
\usepackage{polynom}
\usepackage{tocloft}

\begin{titlepage}
\title{STEP Notes (2011.02.07)}
\author{The Melon Man}
\date{\today}
\end{titlepage}

\renewcommand{\thesection}{\Roman{section}}

% \counterwithout{subsection}{section}  % have subsections numbered independently of sections
% \renewcommand{\thesubsection}{\arabic{subsection}.}  % have dot after subsection number

\addtolength{\cftsecnumwidth}{10pt}

\allowdisplaybreaks

\setlength{\parindent}{0pt}
\setlength{\parskip}{1em}

\begin{document}
\maketitle

I am typesetting my notes for this STEP problem, because I it rather interesting.
Additionally, it is one of the first STEP problems that I have attempted and completed myself -- with almost no aid.
It is a geometric progressions' problem, and a fairly difficult one.
However, I found it simpler than most of the ones that I have seen far.
As I attempt more problems, I will get better at learning how to approach them.
Time-Management skills are not as important now, but they will be developed within due time.

Firstly, it is useful to express the formula for the sum of the first $n$ terms of a geometric progression.
\begin{equation}
    \sum_{i=1}^n ar^{i-1} = a + ar + ar^2 + \cdots + ar^n = \frac{a(1-r^n)}{1-r} \label{Geometric Series}
\end{equation}

We are given some information, all of which is useful.
Two sequences, $a_0, a_1, a_2, \ldots$, and $b_0, b_1, b_2, \ldots$ have the general terms,
\begin{equation*}
    a_n = \lambda^n + \mu^n \quad \text{and} \quad b_n = \lambda^n - \mu^n,
\end{equation*}
where
\begin{equation*}
    \lambda = 1 + \sqrt{2} \quad \text{and} \quad \mu = 1 - \sqrt{2}.
\end{equation*}

Part (i) provides us with a result which we must show is true.
\begin{equation}
    \sum_{r=0}^{n} b_r = -\sqrt{2} + \frac{1}{\sqrt{2}} a_{n+1}
\end{equation}
Additionally, it asks us to give a corresponding result for:
\begin{equation*}
    \sum_{r=0}^{n} a_r
\end{equation*}

Given the information we have, we can write the sum of $b_r$ in terms of $\lambda$ and $\mu$.
\begin{equation*}
    \sum_{r=0}^{n} b_r = \sum_{r=0}^{n} \lambda^r - \mu^r = \sum_{r=0}^{n} \lambda^r - \sum_{r=0}^{n} \mu^r
\end{equation*}
Now we have the difference between two geometric series; both of them can be evaluated using \eqref{Geometric Series}.
The initial term for both sums is 1 ($\lambda^0=\mu^0=1$), and the common ratios are $\lambda$ and $\mu$ respectively.
\begin{align}
    \sum_{i=0}^{n} \lambda^i = \sum_{i=1}^{n} \lambda^{i-1} = \frac{1-\lambda^n}{1-\lambda} + \lambda^n \\
    \sum_{i=0}^{n} \mu^i = \sum_{i=1}^{n} \mu^{i-1} = \frac{1-\mu^n}{1-\mu} + \mu^n
\end{align}

With these, we can find the sum of $b_r$.
It is better to not write $\lambda$ and $\mu$ as the expressions they were given as for the sake of keeping our work simple.
Those definitions will be utilised to simplify when possible.
\begin{align*}
    \sum_{r=0}^{n} b_r & = \frac{1-\lambda^n}{1-\lambda} - \frac{1-\mu^n}{1-\mu} + \lambda^n - \mu^n                                      \\
                       & = \frac{(1-\lambda^n)(1-\mu) - (1-\mu^n)(1-\lambda)}{(1-\lambda)(1-\mu)} + \lambda^n - \mu^n                     \\
                       & = \frac{(1-\lambda^n)(1-1+\sqrt{2}) - (1-\mu^n)(1-1-\sqrt{2})}{(1-1-\sqrt{2})(1-1+\sqrt{2})} + \lambda^n - \mu^n \\
                       & = \frac{\sqrt{2}(1-\lambda^n) + \sqrt{2}(1-\mu^n)}{-2} + \lambda^n - \mu^n                                       \\
                       & = -\frac{\sqrt{2}}{2}\bigg(1-\lambda^n+1-\mu^n\bigg) + \lambda^n - \mu^n                                         \\
                       & = -\frac{1}{\sqrt{2}}\bigg(2-\lambda^n-\mu^n\bigg) + \lambda^n - \mu^n                                           \\
                       & = -\sqrt{2} + \frac{1}{\sqrt{2}}\bigg(\lambda^n+\mu^n\bigg) + \lambda^n - \mu^n                                  \\
                       & = -\sqrt{2} + \frac{1}{\sqrt{2}}\bigg(\lambda^n+\mu^n + \sqrt{2}\lambda^n - \sqrt{2}\mu^n\bigg)                  \\
                       & = -\sqrt{2} + \frac{1}{\sqrt{2}}\bigg(\left(1+\sqrt{2}\right)\lambda^n + \left(1-\sqrt{2}\right)\mu^n\bigg)      \\
                       & = -\sqrt{2} + \frac{1}{\sqrt{2}}\bigg(\lambda\lambda^n + \mu\mu^n\bigg)                                          \\
                       & = -\sqrt{2} + \frac{1}{\sqrt{2}}\bigg(\lambda^{n+1} + \mu^{n+1}\bigg)                                            \\
                       & = -\sqrt{2} + \frac{1}{\sqrt{2}}a_{n+1} \qquad \square
\end{align*}

Now, we may do the same for the sum of $a_r$.
\begin{align*}
    \sum_{r=0}^{n} a_r & = \sum_{r=0}^{n} \lambda^r + \sum_{r=0}^{n} \mu^r                                                \\
                       & = \frac{1-\lambda^n}{1-\lambda} + \frac{1-\mu^n}{1-\mu} + \lambda^n + \mu^n                      \\
                       & = \frac{(1-\lambda^n)(1-\mu) + (1-\mu^n)(1-\lambda)}{(1-\lambda)(1-\mu)} + \lambda^n + \mu^n     \\
                       & = \frac{\sqrt{2}(1-\lambda^n) - \sqrt{2}(1-\mu^n)}{-2} + \lambda^n + \mu^n                       \\
                       & = -\frac{\sqrt{2}}{2}\bigg(1-\lambda^n - 1 + \mu^n\bigg) + \lambda^n + \mu^n                     \\
                       & = \frac{1}{\sqrt{2}}\bigg(\lambda^n - \mu^n\bigg) + \lambda^n + \mu^n                            \\
                       & = \frac{1}{\sqrt{2}}\bigg(\lambda^n - \mu^n + \sqrt{2}\lambda^n + \sqrt{2}\mu^n \bigg)           \\
                       & = \frac{1}{\sqrt{2}}\bigg(\left(1+\sqrt{2}\right)\lambda^n - \left(1-\sqrt{2}\right)\mu^n \bigg) \\
                       & = \frac{1}{\sqrt{2}}\bigg(\lambda^{n+1} - \mu^{n+1} \bigg)                                       \\
                       & = \frac{1}{\sqrt{2}}b_{n+1}
\end{align*}

So far, the most difficult part of the problem has been figuring out how to start it.
I needed to realize that I could write my sum as the difference between two geometric series, and the algebra that followed was relatively simple.
Additionally, I had to not make the mistake of expanding $\lambda$ and $\mu$; this would have made everything much more tedious.
As is characteristic of many STEP questions, I was provided with the result and told to show that it is true.
This allowed me to work backwards, which was a crucial strategy to figure out how to manipulate my expression to obtain the given result.

Finding the sum of $a_r$ was much simpler, as we already knew what to do and had the knowledge to simplify as was needed.
Our result allows us to complete the next part of the problem, which would be daunting otherwise.
Part (ii) requires the evaluation of a double summation.
A result is given for odd values of $n$.
\begin{equation}
    \sum_{m=0}^{2n} \left(\sum_{r=0}^{m} a_r\right) = \frac{1}{2} b^2_{n+1}
\end{equation}
The double sum can be turned into a single sum very easily -- just by substituting our result from part (i).
The rest of it is just algebraic manipulation, except for one thing.
We were told to find a result for the sum specifically for odd values of $n$.
This suggests that the parity of $n$ will be involved at some stage of our solution.
It was fairly easy to recognize where this was.

My simplification caused a $-1$ term to appear in the expression, which I rewrote as $\lambda\mu$.
This was something that was just natural to me.
I had already factorized terms to simplify my expression in part (i), and it is something I do fairly often in maths problems.
From that perspective, $\lambda\mu$ is much more useful to me than some constant term (when none of of my other terms are constants).
When raising $\lambda\mu$ to the power of $n$, the parity of $n$ is important.
\begin{equation}
    \lambda\mu^n = (-1)^n =
    \begin{cases}
        -1 = \phantom{-} \lambda\mu  & \text{if } x \text{ is odd}  \\
        \phantom{-}1  =  -\lambda\mu & \text{if } x \text{ is even}
    \end{cases}
\end{equation}

This is all we need to complete part (ii).
\begin{align*}
    \sum_{m=0}^{2n} \left(\sum_{r=0}^{m} a_r\right) & = \frac{1}{\sqrt{2}} \sum_{m=0}^{2n} b_{m+1}                                                                                                                                                   \\
                                                    & = \frac{1}{\sqrt{2}} \sum_{m=0}^{2n} (\lambda^m - \mu^m) + \sqrt{2} (\lambda^m + \mu^m)                                                                                                        \\
                                                    & = \frac{1}{\sqrt{2}} \sum_{m=0}^{2n} b_m + \sqrt{2} a_m                                                                                                                                        \\
                                                    & = \frac{1}{\sqrt{2}} \left( -\sqrt{2} + \frac{1}{\sqrt{2}} a_{2n+1} + \sqrt{2} \left(\frac{1}{\sqrt{2}} b_{2n+1} \right) \right)                                                               \\
                                                    & = -1 + \frac{1}{2} a_{2n+1} + \frac{1}{\sqrt{2}} b_{2n+1}                                                                                                                                      \\
                                                    & = -1 + \frac{1}{2} \lambda^{2n+1} + \frac{1}{2} \mu^{2n+1} + \frac{1}{\sqrt{2}} \lambda^{2n+1} - \frac{1}{\sqrt{2}} \mu^{2n+1}                                                                 \\
                                                    & = \lambda\mu + \frac{1}{2} \lambda\lambda^{2n} + \frac{1}{2} \mu\mu^{2n} + \frac{1}{\sqrt{2}} \lambda\lambda^{2n} - \frac{1}{\sqrt{2}} \mu\mu^{2n}                                             \\
                                                    & = \lambda\mu + \frac{1}{2} \lambda^{2n} + \frac{1}{2} \mu^{2n} + \frac{1}{\sqrt{2}} \lambda^{2n} - \frac{1}{\sqrt{2}} \mu^{2n} + \frac{1}{\sqrt{2}} \lambda^{2n} - \frac{1}{\sqrt{2}} \mu^{2n} \\
                                                    & \phantom{++}\quad + \lambda^{2n} + \mu^{2n}                                                                                                                                                    \\
                                                    & = \lambda\mu + \frac{1}{2} \bigg(\lambda^n - \mu^n \bigg)^2 + \lambda^n\mu^n + \frac{1}{\sqrt{2}} \bigg(\lambda^n + \mu^n \bigg) \bigg(\lambda^n - \mu^n \bigg)                                \\
                                                    & \phantom{++}\quad + \frac{1}{\sqrt{2}} \bigg(\lambda^n + \mu^n \bigg) \bigg(\lambda^n - \mu^n \bigg) + (\lambda^n + \mu^n)^2 - 2\lambda^n\mu^n                                                 \\
                                                    & = \lambda\mu - \lambda^n\mu^n + \frac{1}{2} \bigg(\lambda^n - \mu^n \bigg)^2 + \sqrt{2} (\lambda^n + \mu^n)(\lambda^n - \mu^n)                                                                 \\
                                                    & \phantom{++}\quad + (\lambda^n + \mu^n)^2                                                                                                                                                      \\
                                                    & = \lambda\mu - \lambda^n\mu^n + \frac{1}{2} b^2_n + \sqrt{2} a_n b_n + a^2_n                                                                                                                   \\
                                                    & = \frac{1}{2} b^2_n + \sqrt{2} a_n b_n + a^2_n                                                                                                                                                 \\
                                                    & = \frac{1}{2} \bigg(b^2_n + 2\sqrt{2} a_n b_n + 2a^2_n \bigg)                                                                                                                                  \\
                                                    & = \frac{1}{2} \bigg(b_n + \sqrt{2} a_n \bigg)^2                                                                                                                                                \\
                                                    & = \frac{1}{2} b^2_{n+1}, \text{ if $n$ is odd}. \qquad \square
\end{align*}

We can do this for even values of $n$, and most of our work can be reused.
\begin{align*}
    \sum_{m=0}^{2n} \left(\sum_{r=0}^{m} a_r\right) & = \lambda\mu - \lambda^n\mu^n + \frac{1}{2} b^2_n + \sqrt{2} a_n b_n + a^2_n \\
                                                    & = \frac{1}{2} b^2_n + \sqrt{2} a_n b_n + a^2_n + 2\lambda^n\mu^n             \\
                                                    & = \frac{1}{2} b^2_{n+1} - 2, \text{ if $n$ is even}.
\end{align*}

\end{document}