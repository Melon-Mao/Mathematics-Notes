\documentclass[12pt]{article}

\usepackage{amsmath}
\usepackage{amssymb}
\usepackage{amsfonts}
\usepackage{hyperref}
\usepackage[style=iso]{datetime2}
\usepackage{xcolor}
\usepackage{polynom}
\usepackage{tocloft}

\begin{titlepage}
\title{STEP Notes (2011.02.07)}
\author{The Melon Man}
\date{\today}
\end{titlepage}

\renewcommand{\thesection}{\Roman{section}}

% \counterwithout{subsection}{section}  % have subsections numbered independently of sections
% \renewcommand{\thesubsection}{\arabic{subsection}.}  % have dot after subsection number

\addtolength{\cftsecnumwidth}{10pt}

\allowdisplaybreaks

\setlength{\parindent}{0pt}
\setlength{\parskip}{1em}

\begin{document}
\maketitle

I am typesetting my notes for this STEP problem, because I it rather interesting.
Additionally, it is one of the first STEP problems that I have attempted and completed myself -- with almost no aid.
It is a geometric progressions' problem, and a fairly difficult one.
However, I found it simpler than most of the ones that I have seen far.
As I attempt more problems, I will get better at learning how to approach them.
Time-Management skills are not as important now, but they will be developed within due time.

Firstly, it is useful to express the formula for the sum of the first $n$ terms of a geometric progression.
\begin{equation}
    \sum_{i=1}^n ar^{i-1} = a + ar + ar^2 + \cdots + ar^n = \frac{a(1-r^n)}{1-r} \label{Geometric Series}
\end{equation}

We are given some information, all of which is useful.
Two sequences, $a_0, a_1, a_2, \ldots$, and $b_0, b_1, b_2, \ldots$ have the general terms,
\begin{equation*}
    a_n = \lambda^n + \mu^n \quad \text{and} \quad b_n = \lambda^n - \mu^n,
\end{equation*}
where
\begin{equation*}
    \lambda = 1 + \sqrt{2} \quad \text{and} \quad \mu = 1 - \sqrt{2}.
\end{equation*}

Part (i) provides us with a result which we must show is true.
\begin{equation}
    \sum_{r=0}^{n} b_r = -\sqrt{2} + \frac{1}{\sqrt{2}} a_{n+1}
\end{equation}
Additionally, it asks us to give a corresponding result for:
\begin{equation*}
    \sum_{r=0}^{n} a_r
\end{equation*}

Given the information we have, we can write the sum of $b_r$ in terms of $\lambda$ and $\mu$.
\begin{equation*}
    \sum_{r=0}^{n} b_r = \sum_{r=0}^{n} \lambda^r - \mu^r = \sum_{r=0}^{n} \lambda^r - \sum_{r=0}^{n} \mu^r
\end{equation*}
Now we have the difference between two geometric series; both of them can be evaluated using \eqref{Geometric Series}.
The initial term for both sums is 1 ($\lambda^0=\mu^0=1$), and the common ratios are $\lambda$ and $\mu$ respectively.
\begin{align}
    \sum_{i=0}^{n} \lambda^i = \sum_{i=1}^{n} \lambda^{i-1} = \frac{1-\lambda^n}{1-\lambda} + \lambda^n \\
    \sum_{i=0}^{n} \mu^i = \sum_{i=1}^{n} \mu^{i-1} = \frac{1-\mu^n}{1-\mu} + \mu^n
\end{align}
With these, we can find the sum of $b_r$.
It is better not write $\lambda$ and $\mu$ as the expressions they were given as for the sake of keeping our work simple.
Those definitions will be utilised to simplify when possible.
\begin{align*}
    \sum_{r=0}^{n} b_r & = \frac{1-\lambda^n}{1-\lambda} - \frac{1-\mu^n}{1-\mu} + \lambda^n - \mu^n                                      \\
                       & = \frac{(1-\lambda^n)(1-\mu) - (1-\mu^n)(1-\lambda)}{(1-\lambda)(1-\mu)} + \lambda^n - \mu^n                     \\
                       & = \frac{(1-\lambda^n)(1-1+\sqrt{2}) - (1-\mu^n)(1-1-\sqrt{2})}{(1-1-\sqrt{2})(1-1+\sqrt{2})} + \lambda^n - \mu^n \\
                       & = \frac{\sqrt{2}(1-\lambda^n) + \sqrt{2}(1-\mu^n)}{-2} + \lambda^n - \mu^n                                       \\
                       & = -\frac{\sqrt{2}}{2}\bigg(1-\lambda^n+1-\mu^n\bigg) + \lambda^n - \mu^n                                         \\
                       & = -\frac{1}{\sqrt{2}}\bigg(2-\lambda^n-\mu^n\bigg) + \lambda^n - \mu^n                                           \\
                       & = -\sqrt{2} + \frac{1}{\sqrt{2}}\bigg(\lambda^n+\mu^n\bigg) + \lambda^n - \mu^n                                  \\
                       & = -\sqrt{2} + \frac{1}{\sqrt{2}}\bigg(\lambda^n+\mu^n + \sqrt{2}\lambda^n - \sqrt{2}\mu^n\bigg)                  \\
                       & = -\sqrt{2} + \frac{1}{\sqrt{2}}\bigg(\left(1+\sqrt{2}\right)\lambda^n + \left(1-\sqrt{2}\right)\mu^n\bigg)      \\
                       & = -\sqrt{2} + \frac{1}{\sqrt{2}}\bigg(\lambda\lambda^n + \mu\mu^n\bigg)                                          \\
                       & = -\sqrt{2} + \frac{1}{\sqrt{2}}\bigg(\lambda^{n+1} + \mu^{n+1}\bigg)                                            \\
                       & = -\sqrt{2} + \frac{1}{\sqrt{2}}a_{n+1} \qquad \square
\end{align*}

Now, we may do the same for the sum of $a_r$.
\begin{align*}
    \sum_{r=0}^{n} a_r & = \sum_{r=0}^{n} \lambda^r + \sum_{r=0}^{n} \mu^r                                            \\
                       & = \frac{1-\lambda^n}{1-\lambda} + \frac{1-\mu^n}{1-\mu} + \lambda^n + \mu^n                  \\
                       & = \frac{(1-\lambda^n)(1-\mu) + (1-\mu^n)(1-\lambda)}{(1-\lambda)(1-\mu)} + \lambda^n + \mu^n \\
                       & = \frac{\sqrt{2}(1-\lambda^n) - \sqrt{2}(1-\mu^n)}{-2} + \lambda^n + \mu^n                   \\
\end{align*}
\end{document}