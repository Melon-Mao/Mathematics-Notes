\documentclass[12pt]{article}

\usepackage{amsmath}
\usepackage{amssymb}
\usepackage{amsfonts}
\usepackage[style=iso]{datetime2}
\usepackage[explicit]{titlesec}
\usepackage{amsthm}

\theoremstyle{definition}
\newtheorem{problem}{Problem}

\begin{titlepage}
\title{Calculus I: Review of Solving Exponential and Logarithmic Equations}
\author{The Melon Man}
\date{\today}
\end{titlepage}

\renewcommand{\thesection}{\Roman{section}}

\allowdisplaybreaks

\setlength{\parindent}{0pt}
\setlength{\parskip}{1em}

\begin{document}
\maketitle

\section{Introduction}
Exponential equations are ones that contain exponential functions discussed in an earlier section.
Logarithmic equations, as the name suggests contain logarithmic functions.
Logarithmic functions are the inverses of exponential functions.
As we would use inverse trigonometric functions to solve trig equations, we would use logarithmic functions to solve exponential equations and vice versa.

We will solve some exponential equations first.
The main property that will be used is:

\begin{equation}
    \log_b(b^x) = x
\end{equation}

This represents the inverse relationship between exponential and logarithmic functions.
Note that we will give our final answers with four decimal place accuracy, if a calculator is used.

\section{Exponential Equations}
Let's solve the following equation:

\begin{problem}
Solve:
\begin{equation*}
    7 + 15e^{1-3z} = 10 \label{eq:1}
\end{equation*}
\end{problem}

We will first rearrange the equation to get the exponential function on its own.

\begin{equation}
    e^{1-3z} = \frac{1}{5}
\end{equation}

We will take the natural logarithm of both sides of our equation and use the property above to simplify.

\begin{align}
    \ln(e^{1-3z}) & = \ln\left(\frac{1}{5}\right)             \\
    1-3z          & = \ln\left(\frac{1}{5}\right)             \\
    3z            & = 1-\ln\left(\frac{1}{5}\right)           \\
    z             & = \frac{1-\ln\left(\frac{1}{5}\right)}{3} \\
    z             & = 0.8698
\end{align}

By taking the natural logarithm, we were able to simplify to a linear equation with $z$ that we were easily able to solve.
Our final answer to Problem~\eqref{eq:1} is:

\begin{equation}
    z = 0.8698
\end{equation}

Let's do another exponential equation.

\begin{problem}
Solve:
\begin{equation*}
    10^{t^2-t} = 100 \label{eq:2}
\end{equation*}
\end{problem}

We may take the common logarithm of both sides.

\begin{align}
    log(10^{t^2-t}) & = log(100) \\
    t^2-t           & = 2
\end{align}

Then we see that we are left with a very simple quadratic that we may solve as follows:

\begin{align}
    t^2-t-2    & = 0  \\
    (t+1)(t-2) & = 0  \\
    \nonumber         \\
    t          & = -1 \\
               & \&   \\
    t          & = 2
\end{align}

Once again, that was rather simple to solve after getting rid of the exponential function.
Our final set of solutions $S$ to Problem~\eqref{eq:2} is:

\begin{equation}
    S = {-1, 2}
\end{equation}

The equations so far have solved only have the variable as part of the exponent.
Let's solve an equation where it is both in the exponent and in other terms.

\begin{problem}
Solve:
\begin{equation*}
    x - xe^{5x+2} = 0 \label{eq:3}
\end{equation*}
\end{problem}

This one is rather simple as we may factor $x$ out of both the terms on the left side of the equation.
Then, we end up with:

\begin{equation}
    x(1-e^{5x+2}) = 0
\end{equation}

It is obvious from here that $x=0$ is one of the solutions to the equation.
We may rearrange and solve for the other solution as follows:

\begin{align}
    1-e^{5x+2}    & = 0            \\
    e^{5x+2}      & = 1            \\
    \ln(e^{5x+2}) & = \ln(1)       \\
    5x + 2        & = 0            \\
    x             & = -\frac{2}{5}
\end{align}

Above, we also used the fact that the logarithm of 1 will be 0 as any number raised to the 0th power is 1.
Also note that dividing $x$ from both sides of the equation at the beginning instead of factorising would have lost the solution $x=0$.
Our solution set $S$ to Problem~\eqref{eq:3} is then:

\begin{equation}
    S = \left\{0, -\frac{2}{5} \right\}
\end{equation}

The following equation looks more complicated but can be solved in relatively the same way as the previous one.

\begin{problem}
Solve:
\begin{equation*}
    5(x^2-4) = (x^2-4)e^{7-x} \label{eq:4}
\end{equation*}
\end{problem}

Again, we should remember to not divide out a term with the variable in to avoid losing a solution.
Instead, we will move everything to one side and factor.
Note that if we are able to divide a term out of an equation, we are also able to factor it out (if the equation is in the right form).

\begin{align}
    5(x^2-4) - (x^2-4)e^{7-x} & = 0 \\
    (x^2-4)(5-e^{7-x})        & = 0
\end{align}

The quadratic $x^2-4$ has the solutions $x=2$ and $x-2$.
The other factor may be solved as follows:

\begin{align}
    5-e^{7-x}    & = 0        \\
    e^{7-x}      & = 5        \\
    \ln(e^{7-x}) & = \ln(5)   \\
    7-x          & = \ln(5)   \\
    x            & = 7-\ln(5) \\
    x            & = 5.3906
\end{align}

Our solution set $S$ to Problem~\eqref{eq:4} is:

\begin{equation}
    S = {2, -2, 5.3906}
\end{equation}

Let's look at a final example, which has multiple exponential terms.

\begin{problem}
Solve:
\begin{equation*}
    4e^{1+3x} - 9e^{5-2x} = 0 \label{eq:5}
\end{equation*}
\end{problem}

We are not able to take the logarithm of both sides as the logarithm of 0 is undefined.
Instead, we will have one exponential on both sides and then take the natural logarithm of both sides.
Then, we may solve as follows:

\begin{align}
    4e^{1+3x}              & = 9e^{5-2x}                               \\
    \ln(4e^{1+3x})         & = \ln(9e^{5-2x})                          \\
    \ln(4) + \ln(e^{1+3x}) & = \ln(9) + \ln(e^{5-2x})                  \\
    \ln(4) + 1+3x          & = \ln(9) + 5-2x                           \\
    5x                     & = \ln(9) + 4 - \ln(4)                     \\
    5x                     & = \ln\left(\frac{9}{4}\right)+4           \\
    x                      & = \frac{\ln\left(\frac{9}{4}\right)+4}{5} \\
    x                      & = 0.9622
\end{align}

We could have also wrote the equation as a quotient of exponential terms and then simplified.
Either way, our final answer to Problem~\eqref{eq:5} is:

\begin{equation}
    x = 0.9622
\end{equation}

\section{Logarithmic Equations}
We will now solve equations with logarithmic terms.
The main property used for this will be:

\begin{equation}
    b^{\log_b(x)} = x
\end{equation}

\begin{problem}
Solve:
\begin{equation*}
    3 + 2\ln\left(\frac{x}{7}+3\right) = -4 \label{eq:6}
\end{equation*}
\end{problem}

We will rearrange to get the logartihmic term on one side and everything else on the rest.

\begin{equation}
    \ln\left(\frac{x}{7}+3\right) = -\frac{7}{2}
\end{equation}

Now, we will raise $e$ to both sides of the equation to cancel out the logarithm.

\begin{align}
    e^{\ln\left(\frac{x}{7}+3\right)} & = e^{-\frac{7}{2}}      \\
    \frac{x}{7}+3                     & = e^{-\frac{7}{2}}      \\
    x                                 & = 7(e^{-\frac{7}{2}}-3) \\
    x                                 & = -20.7889
\end{align}

Now, we cannot move on yet.
Recall that logarithmic functions cannot take in negative arguments.
Our solution begin negative doesn't necessarily mean the argument will be negative.
Putting our solution into the original logarithm will show that it will work, meaning that the solution to Problem~\eqref{eq:6} is:

\begin{equation}
    x = -20.7889
\end{equation}

Let's do a more complciated logarithmic equation.
This will require us to manipulate the equation a bit first as there are multiple logarithmic terms.
We can do so with the following properties of logarithms:

\begin{enumerate}
    \item $\log_b(xy) = \log_b(x) + \log_b(y)$
    \item $\log_b\left(\frac{x}{y}\right) = \log_b(x) - \log_b(y)$
    \item $\log_b(x^y) = y\log_b(x)$
\end{enumerate}

\begin{problem}
Solve:
\begin{equation*}
    2\ln(\sqrt{x}) - \ln(1-x) = 2 \label{eq:7}
\end{equation*}
\end{problem}

We will solve as follows:

\begin{align}
    2\ln(\sqrt{x}) - \ln(1-x)         & = 2                 \\
    \ln((\sqrt{x})^2) - \ln(1-x)      & = 2                 \\
    \ln(x) - \ln(1-x)                 & = 2                 \\
    \ln\left(\frac{x}{1-x}\right)     & = 2                 \\
    e^{\ln\left(\frac{x}{1-x}\right)} & = e^2               \\
    \frac{x}{1-x}                     & = e^2               \\
    x                                 & = (1-x)e^2          \\
    x                                 & = e^2 - e^2x        \\
    x + e^2x                          & = e^2               \\
    x(1+e^2)                          & = e^2               \\
    x                                 & = \frac{e^2}{1+e^2} \\
    x                                 & = 0.8808
\end{align}

This solution is valid for both of our original logarithms so our final answer to Problem~\eqref{eq:7} is:

\begin{equation}
    x = 0.8808
\end{equation}

Let's look at another equation with multiple logarithmic terms.

\begin{problem}
Solve:
\begin{equation*}
    \log(x) + \log(x-3) = 1 \label{eq:8}
\end{equation*}
\end{problem}

We will solve as follows:

\begin{align}
    \log(x) + \log(x-3) & = 1    \\
    \log(x(x-3))        & = 1    \\
    \log(x^2-3x)        & = 1    \\
    10^{\log(x^2-3x)}   & = 10^1 \\
    x^2-3x              & = 10   \\
    x^2-3x-10           & = 0    \\
    (x-5)(x+2)          & = 0    \\
    \nonumber                    \\
    x                   & = 5    \\
                        & \&     \\
    x                   & = -2
\end{align}

Plugging $x=5$ into the original equation, both logarithms will have valid arguments.
However, if we plug in $x=-2$, then both logarithmic terms will have a negative number as their argument.
Since logarithms can only have positive arguments, the only valid solution to Problem~\eqref{eq:8} is:

\begin{equation}
    x = 5
\end{equation}

It is important to plug our solutions back in to the original equation to check if they are valid.
We must make sure it is the original equation we are checking our solutions in and not any of the ones where we have manipulated the logarithms.
If we did, then our solutions would work in them.
The reason is because these manipulations are where the extra solutions which do not work in the original equation are actually introduced.
Additionally, it may not always be the case that we get 2 solutions and only 1 works.
We could have 2 valid solutions, or neither could be valid.

Let's do one more problem.

\begin{problem}
Solve:
\begin{equation*}
    \ln(x-2) + \ln(x+1) = 2 \label{eq:9}
\end{equation*}
\end{problem}

We will solve as follows:

\begin{align}
    \ln(x-2) + \ln(x+1) & = 2                                                 \\
    \ln((x-2)(x+1))     & = 2                                                 \\
    \ln(x^2-x-2)        & = 2                                                 \\
    e^{\ln(x^2-x-2)}    & = e^2                                               \\
    x^2-x-2             & = e^2                                               \\
    x^2-x-(2+e^2)       & = 0                                                 \\
    \nonumber                                                                 \\
    x                   & = \frac{-b\pm\sqrt{b^2-4ac}}{2a}                    \\
    a                   & = 1                                                 \\
    b                   & = -1                                                \\
    c                   & = -(2+e^2)                                          \\
    \nonumber                                                                 \\
    x                   & = \frac{-(-1)\pm\sqrt{(-1)^2-4(1)(-(2+e^2))}}{2(1)} \\
    x                   & = \frac{1\pm\sqrt{1+4(2+e^2)}}{2}                   \\
    x                   & = \frac{1\pm\sqrt{9+4e^2}}{2}                       \\
    \nonumber                                                                 \\
    x                   & = 3.6047                                            \\
                        & \&                                                  \\
    x                   & = - 2.6047
\end{align}

Plugging the two solutions in our original equation, we can see that only $x = 3.6047$ will allow for the logarithms to have valid arguments.
Therefore, the solution to Problem~\eqref{eq:9} is:

\begin{equation}
    x = 3.6047
\end{equation}

\end{document}
