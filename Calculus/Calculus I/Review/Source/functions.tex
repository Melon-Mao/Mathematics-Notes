\documentclass[12pt]{article}

\usepackage{amsmath}
\usepackage{amssymb}
\usepackage{amsfonts}
\usepackage{hyperref}
\usepackage[style=iso]{datetime2}
\usepackage{xcolor}

\begin{titlepage}
\title{Calculus I: Review of Functions}
\author{The Melon Man}
\date{\today}
\end{titlepage}

\renewcommand{\thesection}{\Roman{section}}

\counterwithout{subsection}{section}  % have subsections numbered independently of sections
\renewcommand{\thesubsection}{\arabic{subsection}.}  % have dot after subsection number

\allowdisplaybreaks

\setlength{\parindent}{0pt}
\setlength{\parskip}{1em}

\begin{document}
\maketitle

Functions are incredibly important in calculus.
They will appear in nearly every single section of this course and the later calculus courses.
Therefore, it is important to have a good understanding of what a function is and how to work with them.

What is a function?
A function maps elements from one set to another set.
Specifically, it maps exactly one element from the first set to exactly one element in the second set.
For instance, we may have the mapping $X \mapsto Y$ where $X$ and $Y$ are two sets.
Here, $X$ is the domain of the function and $Y$ is the codomain of the function.
The one-to-one mapping is important to the definition of the function.
Here are two equations in terms of $x$ and $y$:

\begin{align}
    y = x^2 + 1 \\
    y^2 = x + 1 \label{eq:1}
\end{align}

The first equation is a function because for every value of $x$, there is exactly one value of $y$.
Thus, you would call $y$ a function of $x$.
However, the second equation is not a function of $x$ because the exponet is present on $y$.
This may be better illustrated by taking the square root of both sides of equation~\eqref{eq:1}:

\begin{align*}
    y = \pm \sqrt{x + 1}
\end{align*}

We can see that there are two values of $y$ for most values of $x$.
I say most because of instances such as $x = 0$ where there is only one value of $y$ (namely, $y = 1$).
All it takes is one instance of multiple values of $y$ for a given value of $x$ for the equation to not be a function of $x$.
Now, let's look at function notation.

\begin{align}
    \label{eq:2} y = 2x^2-5x+3
\end{align}

Instead of writing it as an equation equal to $y$, we may write some letter with our indepent variable in parentheses.
With this, equation~\eqref{eq:2} becomes $f(x) = 2x^2-5x+3$.
We can use any letter to denote a function, but $f$ is the most common.
If it is already used, then we may use $g$, $h$, or any other letter.

Remember that this does not mean some letter multiplied by $x$.
The parentheses are just a way to denote that the function is a function of $x$.
We may see the usefulness of this if we wished to find the value of our function at a specific value of $x$.
For instance, if we wanted to find the value of $f(x)$ when $x = 2$:

\begin{align*}
    f(2) & = 2(2)^2-5(2)+3 \\
         & = 2(4)-10+3     \\
         & = 8-10+3        \\
         & = 1
\end{align*}

where we were able to replace the $x$ in the parentheses with $2$, as well as all other instances of $x$ in the equation.
We can see how function notation let's us compactly represent functions at some value.
Let's evaluate another function at some points.

Given that $f(x) = -x^2+6x+11$, find the following:

\begin{enumerate}
    \item $f(0)$
    \item $f(2)$
    \item $f(t)$
    \item $f(x-3)$
    \item $f(4x-1)$
\end{enumerate}

The answers are as follows:

\begin{enumerate}
    \item 11
    \item 15
    \item $-t^2+6t+11$
    \item $-(x-3)^2+6(x-3)+11$ = $-x^2+12x+20$
    \item $-(4x-1)^2+6(4x-1)+11$ = $-16x^2+32x+17$
\end{enumerate}

Those were simple enough.
Quite frequently, we want to find the root of a function.
For a function $f(x)$, this is the value of $x$ which satisfies the equation $f(x)=0$.
Functions may have multiple values of $x$ that would satisfy such an equation, so those functions would have multiple roots.
Let's look at an example:

Determine all of the roots of $f(t) = 9t^3 - 18t^2 + 6t$

We would solve this by first setting the function equal to 0:

$$
    9t^3 - 18t^2 + 6t = 0
$$

We would then want to factor the equation as much as possible.
Here, we can factor $t$ out of the left hand side:

$$
    t(9t^2 - 18t + 6) = 0
$$

If either $t$ or $9t^2 + 18t + 6$ is equal to $0$, the whole thing will be $0$.
Thus, we can solve them seperately to get our roots.
We can obviously get $t=0$ as a root.
For the quadratic:

\begin{align*}
    9t^2 - 18t + 6 & = 0                                                \\
    \nonumber                                                           \\
    t              & = \frac{-b\pm\sqrt{b^2-4ac}}{2a}                   \\
    a              & = 9                                                \\
    b              & = -18                                              \\
    c              & = 6                                                \\
    \nonumber                                                           \\
    t              & = \frac{18 \pm \sqrt{18^2-4\cdot9\cdot6}}{2\cdot9} \\
    t              & = \frac{18 \pm \sqrt{108}}{18}                     \\
    t              & = 1 \pm \frac{6\sqrt{3}}{18}                       \\
    t              & = 1 \pm \frac{\sqrt{3}}{3}
\end{align*}

The three roots of our equation are $0$, $1 + \frac{\sqrt{3}}{3}$ and $1 - \frac{\sqrt{3}}{3}$.
That was a rather simple example but we used quite a few important things, such as our factoring skills and the quadratic formula.

Another important thing to note about functions is the domain and range.
As stated before, the domain is any $x$ value we can put in the function that will give us a real value.
The range is any $x$ value that the function can take at all.
Let's practice by finding the domains and ranges of the following functions:

\begin{enumerate}
    \item $f(x)$ = $5x-3$
    \item $g(t)$ = $\sqrt{4-7t}$
    \item $h(x)$ = $-2x^2+12x+5$
    \item $f(z)$ = $|z-6|-3$
    \item $g(x)$ = $8$
\end{enumerate}

For the first one, we notice that the function can take any real value of $x$ and will correspondingly output a real value.
Therefore, it has every real number in its domain. Formally, we would write that the domain of $f(x)$ is $(-\inf, \inf)$

\end{document}