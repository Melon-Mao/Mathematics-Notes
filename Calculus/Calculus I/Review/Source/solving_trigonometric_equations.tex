\documentclass[12pt]{article}

\usepackage{amsmath}
\usepackage{amssymb}
\usepackage{amsfonts}
\usepackage[style=iso]{datetime2}
\usepackage{graphicx}
\usepackage[explicit]{titlesec}
\usepackage{amsthm}

\theoremstyle{definition}
\newtheorem{problem}{Problem}

\graphicspath{ {./Images/} }

\begin{titlepage}
\title{Calculus I: Review of Solving Trigonometric Equations}
\author{The Melon Man}
\date{\today}
\end{titlepage}

\renewcommand{\thesection}{\Roman{section}}

\allowdisplaybreaks

\setlength{\parindent}{0pt}
\setlength{\parskip}{1em}

\begin{document}
\maketitle

Let's solve the following trig equation:

\begin{problem}
Solve:
\begin{equation*}
    {2\cos(t) = \sqrt{3}} \label{eq:1}
\end{equation*}
\end{problem}

We can solve this by first dividing both sides by $2$:

\begin{equation}
    \cos(t) = \frac{\sqrt{3}}{2}
\end{equation}

We have to think of some angle $t$ which the cosine of gives us $\frac{\sqrt{3}}{2}$.
We can use the first quadrant of the unit circle to get the angle $\frac{\pi}{6}$.
The symmetrical nature of the unit circle gives us the angle $\frac{11\pi}{6}$ as another solution.
Note that due to the periodic nature of the cosine function, there will be an infinite amount of solutions to our equation, spaced evenly by the period of the function ($2\pi$).
Our solution to Equation~\eqref{eq:1} is:

\begin{equation}
    t = \frac{\pi}{6} + 2\pi k , \frac{11\pi}{6} + 2\pi k, k \in \mathbb{Z}
\end{equation}

where $\mathbb{Z}$ is the set of integers.

Now, most trig equations that will be solved in a calculus course will be slightly different from this.
We would usually work in intervals of the angle, and find solutions in that interval.
For instance, we may solve the following trig equation:


\begin{problem}
Solve on $\displaystyle [-2\pi, 2\pi]$:
\begin{equation*}
    {2\cos(t) = \sqrt{3}} \label{eq:2}
\end{equation*}
\end{problem}

We already have the general solution for this equation.
To solve on the interval, we just have to plug in values of $k$ into the solution for values that fall in the interval.
With $k=0$, we get $t=\frac{\pi}{6}$ and $t=\frac{11\pi}{6}$.
If we were to add $2\pi$ ($k=1$) to either of those the value of $t$ would be outside of the interval, so we should now look at values of $k$ smaller than $0$.
With $k=-1$, we would get $t=-\frac{11\pi}{6}$ and $t=-\frac{\pi}{6}$.
Once again, we cannot use a smaller value of $k$ as we would go outside of our interval.
Then, the solution $S$ set for Equation~\eqref{eq:2} may be expressed as follows:

\begin{equation}
    S = \left\{ \frac{\pi}{6}, \frac{11\pi}{6}, -\frac{\pi}{6}, -\frac{11\pi}{6} \right\}
\end{equation}

We may apply our skills on another trig equation.

\begin{problem}
Solve on $\displaystyle [-\pi, 2\pi]$:
\begin{equation*}
    2\sin(5x) = -\sqrt{3} \label{eq:3}
\end{equation*}
\end{problem}

We will solve as follows:

\begin{align}
    2\sin(5x) & = -\sqrt{3}                          \\
    \sin(5x)  & = -\frac{\sqrt{3}}{2}                \\
    \nonumber                                        \\
    5x        & = \frac{4\pi}{3} + 2\pi k            \\
              & \&                                   \\
    5x        & = \frac{5\pi}{3} + 2\pi k            \\
    \nonumber                                        \\
    x         & = \frac{4\pi}{15} + \frac{2\pi k}{5} \\
              & \&                                   \\
    x         & = \frac{\pi}{3} + \frac{2\pi k}{5}   \\
    \nonumber                                        \\
    k         & \in \mathbb{Z}
\end{align}

Now we will plug in values of $k$ to get all of our solutions in the given interval.

At $k=0$:
\begin{align}
    x & =\frac{4\pi}{15} + \frac{2(0)\pi}{5} = \frac{4\pi}{15} < 2\pi \\
      & \&                                                            \\
    x & =\frac{\pi}{3} + \frac{2(0)\pi}{5} = \frac{\pi}{3} < 2\pi
\end{align}

At $k=1$:
\begin{align}
    x & =\frac{4\pi}{15} + \frac{2(1)\pi}{5} = \frac{2\pi}{3} < 2\pi \\
      & \&                                                           \\
    x & =\frac{\pi}{3} + \frac{2(1)\pi}{5} = \frac{11\pi}{15} < 2\pi
\end{align}

At $k=2$:
\begin{align}
    x & =\frac{4\pi}{15} + \frac{2(2)\pi}{5} = \frac{16\pi}{15} < 2\pi \\
      & \&                                                             \\
    x & =\frac{\pi}{3} + \frac{2(2)\pi}{5} = \frac{17\pi}{15} < 2\pi
\end{align}

At $k=3$:
\begin{align}
    x & =\frac{4\pi}{15} + \frac{2(3)\pi}{5} = \frac{22\pi}{15} < 2\pi \\
      & \&                                                             \\
    x & =\frac{\pi}{3} + \frac{2(3)\pi}{5} = \frac{23\pi}{15} < 2\pi
\end{align}

At $k=4$:
\begin{align}
    x & =\frac{4\pi}{15} + \frac{2(4)\pi}{5} = \frac{28\pi}{15} < 2\pi \\
      & \&                                                             \\
    x & =\frac{\pi}{3} + \frac{2(4)\pi}{5} = \frac{29\pi}{15} < 2\pi
\end{align}

At $k=5$:
\begin{align}
    x & =\frac{4\pi}{15} + \frac{2(5)\pi}{5} = \frac{34\pi}{15} > 2\pi \\
      & \&                                                             \\
    x & =\frac{\pi}{3} + \frac{2(5)\pi}{5} = \frac{7\pi}{3} > 2\pi
\end{align}

So far, we have found $10$ solutions for the equation.
Now let's look at negative values for $k$.

At $k=-1$:
\begin{align}
    x & =\frac{4\pi}{15} + \frac{2(-1)\pi}{5} = -\frac{2\pi}{15} > -\pi \\
      & \&                                                              \\
    x & =\frac{\pi}{3} + \frac{2(-1)\pi}{5} = -\frac{\pi}{15} > -\pi
\end{align}

At $k=-2$:
\begin{align}
    x & =\frac{4\pi}{15} + \frac{2(-2)\pi}{5} = -\frac{8\pi}{15} > -\pi \\
      & \&                                                              \\
    x & =\frac{\pi}{3} + \frac{2(-2)\pi}{5} = -\frac{7\pi}{15} > -\pi
\end{align}

At $k=-3$:
\begin{align}
    x & =\frac{4\pi}{15} + \frac{2(-3)\pi}{5} = -\frac{14\pi}{15} > -\pi \\
      & \&                                                               \\
    x & =\frac{\pi}{3} + \frac{2(-3)\pi}{5} = -\frac{13\pi}{15} > -\pi
\end{align}

At $k=-4$:
\begin{align}
    x & =\frac{4\pi}{15} + \frac{2(-4)\pi}{5} = -\frac{4\pi}{3} < -\pi \\
      & \&                                                             \\
    x & =\frac{\pi}{3} + \frac{2(-4)\pi}{5} = -\frac{19\pi}{15} < -\pi
\end{align}

Then, we have $16$ solutions in total for Equation~\eqref{eq:3}.
The solution set $S$ to Equation~\eqref{eq:3} is thus:

\begin{equation}
    \begin{aligned}
        S & = \left\{ \frac{4\pi}{15}, \frac{\pi}{3}, \frac{2\pi}{3}, \frac{11\pi}{15}, \frac{16\pi}{15}, \frac{17\pi}{15}, \frac{22\pi}{15}, \frac{23\pi}{15}, \frac{28\pi}{15}, \frac{29\pi}{15}, \right. \\
          & \qquad \left. -\frac{\pi}{15}, -\frac{2\pi}{15}, -\frac{7\pi}{15}, -\frac{8\pi}{15}, -\frac{13\pi}{15}, -\frac{14\pi}{15} \right\}
    \end{aligned}
\end{equation}

Let's do another equation that is a bit more complex.

\begin{problem}
Solve on $\displaystyle \left[-\frac{3\pi}{2}, \frac{3\pi}{2}\right]$:
\begin{equation*}
    \sin(2x) = -\cos(2x) \label{eq:4}
\end{equation*}
\end{problem}

We can rewrite this equation as follows:

\begin{align}
    -\frac{\sin(2x)}{\cos(2x)} & = 1  \\
    -\tan(2x)                  & = 1  \\
    \tan(2x)                   & = -1
\end{align}

Now we have an equation in terms of tangent.
Looking at the unit circle, we want the angle where the sine and cosine differ by a negative sign.
There are two angles opposite to each other that satisfy this.
Those are the angles $\frac{3\pi}{4}$ and $\frac{7\pi}{4}$.
Note that the second is just the first angle summed with $\pi$.
Then, we can see that this repeats every $\pi$ so our general solution is:

\begin{align}
    2x & = \frac{3\pi}{4} + \pi k           \\
    x  & = \frac{3\pi}{8} + \frac{\pi k}{2} \\
    k  & \in \mathbb{Z}
\end{align}

We can see that $x$ at $k=1$ and $k=2$ is $\frac{7\pi}{8}$ and $\frac{11\pi}{8}$ respectively.
Any values of $k$ above that will be outside our interval.
$k=0$ will give us $\frac{3\pi}{8}$.
$k=-1$, $k=2$ and $k=3$ give $-\frac{\pi}{8}$, $-\frac{5\pi}{8}$ and $-\frac{9\pi}{8}$.
That is the last value of $k$ before we go outside our interval, so we can write our solution set $S$ for Equation~\eqref{eq:4} as:

\begin{equation}
    S = \left\{ \frac{3\pi}{8}, \frac{7\pi}{8}, \frac{11\pi}{8}, -\frac{\pi}{8}, -\frac{5\pi}{8}, -\frac{9\pi}{8} \right\}
\end{equation}

The method in which Equation~\eqref{eq:4} was solve is the 'standard' method.
We could have had two sets of solutions and have added $2\pi k$ to both.
This would be more familiar with the method of solving equations in terms of sine and cosines, and would have gotten us the same solutions in the given interval.
For the tangent function, the way we did it is standard and better represents the nature of the function by having one solution that repeats every $\pi$.
Let's do one more problem to show something important about trig equations.


\begin{problem}
Solve:
\begin{equation*}
    \cos(3x) = 2 \label{eq:5}
\end{equation*}
\end{problem}

This equation has no solutions.
This is because the sine and cosine functions do not return values greater than $1$ or less than $-1$.
Therefore, there is no angle that would satisfy the above equation.
This shows that not all trig equations will have solutions.

We have simple trig equations.
There are more complicated trig equations that can be solved, which do not have solutions that are the common angles found on the unit circle.
These equations require a calculator and will below

\end{document}